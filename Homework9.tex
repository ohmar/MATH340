%----------------------------------------------------------------------------------------
%	PACKAGES AND OTHER DOCUMENT CONFIGURATIONS
%----------------------------------------------------------------------------------------

\documentclass{article}

\usepackage{amsfonts}
\usepackage{amsmath}
\usepackage{amsthm}
\usepackage{fancyhdr} % Required for custom headers
\usepackage{lastpage} % Required to determine the last page for the footer
\usepackage{extramarks} % Required for headers and footers
\usepackage{graphicx} % Required to insert images
\usepackage{lipsum} % Used for inserting dummy 'Lorem ipsum' text into the template

% Margins
\topmargin=-0.45in
\evensidemargin=0in
\oddsidemargin=0in
\textwidth=6.5in
\textheight=9.0in
\headsep=0.25in 

\linespread{1.1} % Line spacing

% Set up the header and footer
\pagestyle{fancy}
\lhead{\hmwkAuthorName} % Top left header
\chead{\hmwkClass\ (\hmwkClassInstructor\ \hmwkClassTime): \hmwkTitle} % Top center header
\rhead{\firstxmark} % Top right header
\lfoot{\lastxmark} % Bottom left footer
\cfoot{} % Bottom center footer
\rfoot{Page\ \thepage\ of\ \pageref{LastPage}} % Bottom right footer
\renewcommand\headrulewidth{0.4pt} % Size of the header rule
\renewcommand\footrulewidth{0.4pt} % Size of the footer rule

\setlength\parindent{0pt} % Removes all indentation from paragraphs

%----------------------------------------------------------------------------------------
%	DOCUMENT STRUCTURE COMMANDS
%	Skip this unless you know what you're doing
%----------------------------------------------------------------------------------------

% Header and footer for when a page split occurs within a problem environment
\newcommand{\enterProblemHeader}[1]{
\nobreak\extramarks{#1}{#1 continued on next page\ldots}\nobreak
\nobreak\extramarks{#1 (continued)}{#1 continued on next page\ldots}\nobreak
}

% Header and footer for when a page split occurs between problem environments
\newcommand{\exitProblemHeader}[1]{
\nobreak\extramarks{#1 (continued)}{#1 continued on next page\ldots}\nobreak
\nobreak\extramarks{#1}{}\nobreak
}

\setcounter{secnumdepth}{0} % Removes default section numbers
\newcounter{homeworkProblemCounter} % Creates a counter to keep track of the number of problems

\newcommand{\homeworkProblemName}{}
\newenvironment{homeworkProblem}[1][Problem \arabic{homeworkProblemCounter}]{ % Makes a new environment called homeworkProblem which takes 1 argument (custom name) but the default is "Problem #"
\stepcounter{homeworkProblemCounter} % Increase counter for number of problems
\renewcommand{\homeworkProblemName}{#1} % Assign \homeworkProblemName the name of the problem
\section{\homeworkProblemName} % Make a section in the document with the custom problem count
\enterProblemHeader{\homeworkProblemName} % Header and footer within the environment
}{
\exitProblemHeader{\homeworkProblemName} % Header and footer after the environment
}

\newcommand{\problemAnswer}[1]{ % Defines the problem answer command with the content as the only argument
\noindent\framebox[\columnwidth][c]{\begin{minipage}{0.98\columnwidth}#1\end{minipage}} % Makes the box around the problem answer and puts the content inside
}

\newcommand{\homeworkSectionName}{}
\newenvironment{homeworkSection}[1]{ % New environment for sections within homework problems, takes 1 argument - the name of the section
\renewcommand{\homeworkSectionName}{#1} % Assign \homeworkSectionName to the name of the section from the environment argument
\subsection{\homeworkSectionName} % Make a subsection with the custom name of the subsection
%\enterProblemHeader{\homeworkProblemName\ [\homeworkSectionName]} % Header and footer within the environment
}{
%\enterProblemHeader{\homeworkProblemName} % Header and footer after the environment
}
   
%----------------------------------------------------------------------------------------
%	NAME AND CLASS SECTION
%----------------------------------------------------------------------------------------

\newcommand{\hmwkTitle}{Assignment\ \#6} % Assignment title
\newcommand{\hmwkDueDate}{Monday,\ November\ 13$^{th}$,\ 2013} % Due date
\newcommand{\hmwkClass}{MATH\ 301} % Course/class
\newcommand{\hmwkClassTime}{2:10pm} % Class/lecture time
\newcommand{\hmwkClassInstructor}{Dr. Eric Hayashi} % Teacher/lecturer
\newcommand{\hmwkAuthorName}{Omar Sandoval} % Your name

%----------------------------------------------------------------------------------------
%	TITLE PAGE
%----------------------------------------------------------------------------------------

\title{
\vspace{2in}
\textmd{\textbf{\hmwkClass:\ \hmwkTitle}}\\
\normalsize\vspace{0.1in}\small{Due\ on\ \hmwkDueDate}\\
\vspace{0.1in}\large{\textit{\hmwkClassInstructor\ \hmwkClassTime}}
\vspace{3in}
}

\author{\textbf{\hmwkAuthorName}}
\date{Monday,\ October\ 28$^{th}$,\ 2013} % Insert date here if you want it to appear below your name

%----------------------------------------------------------------------------------------

\begin{document}

\maketitle

%----------------------------------------------------------------------------------------
%	TABLE OF CONTENTS
%----------------------------------------------------------------------------------------

%\setcounter{tocdepth}{1} % Uncomment this line if you don't want subsections listed in the ToC

\newpage
\tableofcontents
\newpage

%----------------------------------------------------------------------------------------
%	PROBLEM 1
%----------------------------------------------------------------------------------------

% To have just one problem per page, simply put a \clearpage after each problem
\begin{homeworkProblem}[Exercise\ 3.10.6]
Suppose that $n$ observations are chosen at random from a continuous pdf, $f_y(y)$, whose median is $m$. Is $P(Y_1' > m)$ less than, equal to, or greater than $P(Y_n' > m)$?

\problemAnswer{
}
\end{homeworkProblem}

%----------------------------------------------------------------------------------------
%	PROBLEM 2
%----------------------------------------------------------------------------------------

\begin{homeworkProblem}[Exercise\ 3.10.8] % Custom section title
A random sample of size $n = 5$ is drawn from the pdf, $f_y(y) = 2y, 0 \le y \le 1$. On the same set of axes, graph the pdfs from $Y_2$, $Y_1'$, and $Y_5'$.

\problemAnswer{
}
\end{homeworkProblem}

%----------------------------------------------------------------------------------------
%	PROBLEM 3
%----------------------------------------------------------------------------------------

\begin{homeworkProblem}[Exercise\ 3.10.10] % Custom section title
Suppose that $n$ observations are chosen at random from a continuous pdf $f_y(y)$. What is the probability that the last observation recorded will be the smallest number in the entire sample?

\problemAnswer{
}
\end{homeworkProblem}

%----------------------------------------------------------------------------------------
%	PROBLEM 4
%----------------------------------------------------------------------------------------
\begin{homeworkProblem}[Exercise\ 3.11.6]
Let $X$ denote the number on a chip drawn at random from an urn containing three chips, numbered 1, 2, and 3. Let $Y$ be the number of heads that occur when a fair coin is tossed $X$ times.

\begin{homeworkSection}{(a) Find $p_{x,y}(x,y)$.}
\problemAnswer{
}
\begin{homeworkSection}{(b) Find the marginal pdf of $Y$ by summing out the $x$ values.}
\problemAnswer{
}
\end{homeworkSection}
\end{homeworkSection}
\end{homeworkProblem}
%----------------------------------------------------------------------------------------
%	PROBLEM 5
%----------------------------------------------------------------------------------------
\begin{homeworkProblem}[Exercise\ 3.11.9]
Let $X$ and $Y$ be independent random variables where $p_x(k) = e^{\lambda(\frac{\lambda^k}{k!})}$ and $p_y(k) = e^{-\mu(\frac{\mu^k}{k!})}$ for $k = 0, 1, ...$. Show that the conditional pdf of $X$ given that $X + Y = n$ is binomial with parameters $n$ and $\frac{\lambda}{\lambda + \mu}$. (Hint: See Question 3.8.1)

\problemAnswer{
}
\end{homeworkProblem}

%----------------------------------------------------------------------------------------
%	PROBLEM 6
%----------------------------------------------------------------------------------------
\begin{homeworkProblem}[Exercise\ 3.11.12]
Given the joing pdf,
\begin{align*}
	f_{X,Y}(x,y) = 2e^-{x+y},	 0 \le x \le y,		 y \ge 0
\end{align*}
\begin{homeworkSection}{}
\problemAnswer{
}
\begin{homeworkSection}{}
\problemAnswer{
}
\begin{homeworkSection}{}
\problemAnswer{
}
\begin{homeworkSection}{}
\problemAnswer{
}
\end{homeworkSection}
\end{homeworkSection}
\end{homeworkSection}
\end{homeworkSection}
\end{homeworkProblem}
%----------------------------------------------------------------------------------------
%	PROBLEM 7
%----------------------------------------------------------------------------------------
\begin{homeworkProblem}[Problem\ 7]
\begin{homeworkSection}{}
\problemAnswer{
}
\begin{homeworkSection}{}
\problemAnswer{
}
\begin{homeworkSection}{}
\problemAnswer{
}
\end{homeworkSection}
\end{homeworkSection}
\end{homeworkSection}
\end{homeworkProblem}
\end{document}
